%%%%%%%%%%%%%%%%%%%%%%%%%%%%%%%%%%%%%%%%%
% Structured General Purpose Assignment
% LaTeX Template
%
% This template has been downloaded from:
% http://www.latextemplates.com
%
% Original author:
% Ted Pavlic (http://www.tedpavlic.com)
%
% Note:
% The \lipsum[#] commands throughout this template generate dummy text
% to fill the template out. These commands should all be removed when 
% writing assignment content.
%
%%%%%%%%%%%%%%%%%%%%%%%%%%%%%%%%%%%%%%%%%

%----------------------------------------------------------------------------------------
%	PACKAGES AND OTHER DOCUMENT CONFIGURATIONS
%----------------------------------------------------------------------------------------

\documentclass{article}


\usepackage{listings}
\lstset{ 
        language=Matlab,                                % choose the language of the code
%       basicstyle=10pt,                                % the size of the fonts that are used for the code
        %numbers=left,                                   % where to put the line-numbers
        %numberstyle=\footnotesize,                      % the size of the fonts that are used for the line-numbers
        %stepnumber=5,                                           % the step between two line-numbers. If it's 1 each line will be numbered
        %numbersep=5pt,                                  % how far the line-numbers are from the code
		%backgroundcolor=\color{light-gray},          % choose the background color. You must add \usepackage{color}
        showspaces=false,                               % show spaces adding particular underscores
        showstringspaces=false,                         % underline spaces within strings
        showtabs=false,                                         % show tabs within strings adding particular underscores
%       frame=single,                                           % adds a frame around the code
%       tabsize=2,                                              % sets default tabsize to 2 spaces
%       captionpos=b,                                           % sets the caption-position to bottom
        breaklines=true,                                        % sets automatic line breaking
        breakatwhitespace=false,                        % sets if automatic breaks should only happen at whitespace
        escapeinside={\%*}{*)},                          % if you want to add a comment within your code
        xleftmargin=.50in
}
\usepackage{color}
\usepackage{fancyhdr} % Required for custom headers
\usepackage{lastpage} % Required to determine the last page for the footer
\usepackage{extramarks} % Required for headers and footers
\usepackage{graphicx} % Required to insert images
\usepackage{lipsum} % Used for inserting dummy 'Lorem ipsum' text into the template
\usepackage{MnSymbol}% Used for \Box






% Margins
\topmargin=-0.45in
\evensidemargin=0in
\oddsidemargin=0in
\textwidth=6.5in
\textheight=9.0in
\headsep=0.25in 

\linespread{1.1} % Line spacing

% Set up the header and footer
\pagestyle{fancy}
\lhead{\hmwkAuthorName} % Top left header
\chead{\hmwkClass\ \hmwkTitle} % Top center header
\rhead{\firstxmark} % Top right header
\lfoot{\lastxmark} % Bottom left footer
\cfoot{} % Bottom center footer
\rfoot{Page\ \thepage\ of\ \pageref{LastPage}} % Bottom right footer
\renewcommand\headrulewidth{0.4pt} % Size of the header rule
\renewcommand\footrulewidth{0.4pt} % Size of the footer rule

\setlength\parindent{0pt} % Removes all indentation from paragraphs

%----------------------------------------------------------------------------------------
%	DOCUMENT STRUCTURE COMMANDS
%	Skip this unless you know what you're doing
%----------------------------------------------------------------------------------------

% Header and footer for when a page split occurs within a problem environment
\newcommand{\enterProblemHeader}[1]{
\nobreak\extramarks{#1}{#1 continued on next page\ldots}\nobreak
\nobreak\extramarks{#1 (continued)}{#1 continued on next page\ldots}\nobreak
}

% Header and footer for when a page split occurs between problem environments
\newcommand{\exitProblemHeader}[1]{
\nobreak\extramarks{#1 (continued)}{#1 continued on next page\ldots}\nobreak
\nobreak\extramarks{#1}{}\nobreak
}

\setcounter{secnumdepth}{0} % Removes default section numbers
\newcounter{homeworkProblemCounter} % Creates a counter to keep track of the number of problems

\newcommand{\homeworkProblemName}{}
\newenvironment{homeworkProblem}[1][Problem \arabic{homeworkProblemCounter}]{ % Makes a new environment called homeworkProblem which takes 1 argument (custom name) but the default is "Problem #"
\stepcounter{homeworkProblemCounter} % Increase counter for number of problems
\renewcommand{\homeworkProblemName}{#1} % Assign \homeworkProblemName the name of the problem
\section{\homeworkProblemName} % Make a section in the document with the custom problem count
\enterProblemHeader{\homeworkProblemName} % Header and footer within the environment
}{
\exitProblemHeader{\homeworkProblemName} % Header and footer after the environment
}

\newcommand{\problemAnswer}[1]{ % Defines the problem answer command with the content as the only argument
\noindent\framebox[\columnwidth][c]{\begin{minipage}{0.98\columnwidth}#1\end{minipage}} % Makes the box around the problem answer and puts the content inside
}

\newcommand{\homeworkSectionName}{}
\newenvironment{homeworkSection}[1]{ % New environment for sections within homework problems, takes 1 argument - the name of the section
\renewcommand{\homeworkSectionName}{#1} % Assign \homeworkSectionName to the name of the section from the environment argument
\subsection{\homeworkSectionName} % Make a subsection with the custom name of the subsection
\enterProblemHeader{\homeworkProblemName\ [\homeworkSectionName]} % Header and footer within the environment
}{
\enterProblemHeader{\homeworkProblemName} % Header and footer after the environment
}
   
%----------------------------------------------------------------------------------------
%	NAME AND CLASS SECTION
%----------------------------------------------------------------------------------------

\newcommand{\hmwkTitle}{Assignment\ 2} % Assignment title
\newcommand{\hmwkDueDate}{Tue,\ Oct.\ 15,\ 2013} % Due date
\newcommand{\hmwkClass}{CS\ 444} % Course/class
\newcommand{\hmwkClassTime}{} % Class/lecture time
\newcommand{\hmwkClassInstructor}{} % Teacher/lecturer
\newcommand{\hmwkAuthorName}{Qiyuan Qiu} % Your name

%----------------------------------------------------------------------------------------
%	TITLE PAGE
%----------------------------------------------------------------------------------------

\title{
\vspace{2in}
\textmd{\textbf{\hmwkClass:\ \hmwkTitle}}\\
\normalsize\vspace{0.1in}\small{Due\ on\ \hmwkDueDate}\\
\vspace{0.1in}\large{\textit{\hmwkClassInstructor\ \hmwkClassTime}}
\vspace{3in}
}

\author{\textbf{\hmwkAuthorName}}
\date{} % Insert date here if you want it to appear below your name

%----------------------------------------------------------------------------------------

\begin{document}

\maketitle

%----------------------------------------------------------------------------------------
%	TABLE OF CONTENTS
%----------------------------------------------------------------------------------------

%\setcounter{tocdepth}{1} % Uncomment this line if you don't want subsections listed in the ToC

%\newpage
%\tableofcontents

\newpage

%----------------------------------------------------------------------------------------
%	PROBLEM 1
%----------------------------------------------------------------------------------------

% To have just one problem per page, simply put a \clearpage after each problem

\begin{homeworkProblem}
\vspace{10pt} % Question
    Validity
\begin{homeworkSection}
{(a) $\models \phi \Rightarrow (\psi \Rightarrow \phi)$}

\problemAnswer{ % This here will give you a box sourrounding your answer.
For any model $M=(D,I)$ \\
    $\phi \Rightarrow (\psi \Rightarrow \phi)$ iff \\
    $\neg \phi \vee (\psi \Rightarrow \phi)$ iff \\
    $\neg \phi \vee \neg \psi \vee \phi$ \\
    Because $\neg \ \phi \vee \ \phi$ is always true, therefore the original formula is always true with  \\
    regard to all models. 
}
\end{homeworkSection}

\begin{homeworkSection}
{(b) $\models (P(A) \wedge A=B) \Rightarrow P(B)$}

\problemAnswer{ % This here will give you a box sourrounding your answer.
For any model $M=(D,I)$ \\
    $(P(A) \wedge A=B) \Rightarrow P(B)$ iff \\
    $\neg (P(A) \wedge A=B) \vee P(B)$ iff \\ 
    $\neg P(A) \vee \neg (A=B) \vee P(B)$ \\
    There are two cases:\\
    First, when A=B is true,  $\neg P(A) \vee \neg (A=B) \vee P(B)$ is equivalent to $\neg P(A) \vee P(A)$, which is always true.\\
    Second, when A=B is false, $\neg (A=B)$ is true, then the original formula is true with reagard to any model. 
}
\end{homeworkSection}

\begin{homeworkSection}
{(c) $\models (\exists x A = x)$}

\problemAnswer{ % This here will give you a box sourrounding your answer.
\begin{tabular}{l l}
$\models (\exists x A = x)$ iff for all models $M=(D,I)$ $\models_{M} A=x$ & Truth in all models\\
iff for all v.a. $U$ and all models $M$\\
$\models_{M}(\exists x A = x)[U]$ & Satisfaction condition \\
iff for some $d\in D $ $T_{IU_{x:d}} \in (A=x)^I = \{A\}$  & Term assignment \\
iff for some $d \in \{A\} \cup \{D-\{A\}\} , d \in \{A\}$ & Definition of term assignment \\
TRUE
\end{tabular}
}
\end{homeworkSection}

\begin{homeworkSection}
{(d) $\models (\forall x (\forall y(( P(x) \wedge x = y) \Rightarrow P(y))))$}

\problemAnswer{ % This here will give you a box sourrounding your answer.
Let $M = (D,I) be an arbitrary model$, let $d1, d2\in D \wedge \neg d1 = d2$ be two arbitrary varialbles. \\
The original claim is true iff\\
\begin{tabular}{l l}
iff $\models_{M}(\forall x (\forall y(( P(x) \wedge x = y) \Rightarrow P(y))))[U]$ & Definition of truth in a model\\
iff $\models_{M}(\forall x (\forall y(( P(x) \wedge x = y) \Rightarrow P(y))))[U_{x:d_{1}\ y:d_{2}}]$ & Definition of satisfaction condition for $\forall$ \\
iff for all $<d_{1},d_{2}> \in D^2$ $T_{IU_{x:d{1}\ y:d_{2}}} \in (P(X) \wedge x = y \Rightarrow P(y))^I$ &Term assignment \\
iff for all $<d_{1},d_{2}> \in D^2$ , $<d_{1},d_{2}> \in D^2$ & Definition of variable assignment \\
TRUE
\end{tabular}
}
\end{homeworkSection}

\end{homeworkProblem}

%----------------------------------------------------------------------------------------
%	PROBLEM 2
%----------------------------------------------------------------------------------------

% To have just one problem per page, simply put a \clearpage after each problem

\begin{homeworkProblem}
\vspace{10pt} % Question
\begin{homeworkSection}
{(a) Prove that $P(A), A=B \models P(B)$}
\problemAnswer{ % This here will give you a box sourrounding your answer.
    For all model $M = (D,I)$, any v.a. $U$, \\
    $\models_{M} P(A), A=B [U]$ \\
    $T_{IU}(A) \in P^I$\\
    $T_{IU}(A)  = T_{IU}(B)$\\
    Therefore, \\
    $T_{IU}(B) = T_{IU}(A) \in P^I$\\
    $T_{IU}(B) \in P^I$
}
\end{homeworkSection}

\begin{homeworkSection}
{(b) Prove that $\phi \models (\psi \Rightarrow \phi)$}
\problemAnswer{ % This here will give you a box sourrounding your answer.
For all $M$, all v.a. $U$ \\
$\models_{M} \phi [U]$ \\
$\models_{M} \psi \Rightarrow \phi [U]$ \\
iff $\models_{M} \neg \psi \Rightarrow \phi [U]$ \\
iff $\neg \models_{M} \psi [U] \vee \models_{M} \phi [U] $  which is always true.\\
Therefore $\phi \models (\phi \Rightarrow \psi)$
}
\end{homeworkSection}

\begin{homeworkSection}
{(c) Prove that $\frac{\phi}{\phi \vee \psi}$ is sound}
\problemAnswer{ % This here will give you a box sourrounding your answer.
For all $M$, all v.a. $U$ \\
$\models_{M} \phi [U]$ \\
This is equivalent to\\
$\models_{M} \phi [U] \vee \models_{M}\psi [U]$\\
All models of $\phi$ are models of $\phi \vee \psi$.\\
Therefore this is sound.
}
\end{homeworkSection}

\begin{homeworkSection}
{(d) Prove that $\frac{\phi \Rightarrow \psi , \psi \Rightarrow \chi}{\phi \vee \psi} $ is sound}
\problemAnswer{ % This here will give you a box sourrounding your answer.
For all $M$, all v.a. $U$ \\
$\models_{M} \Rightarrow \psi [U]$ iff $\models_{M} \neg \phi [U] \vee \models_{M} \psi [U]$\\
$\models_{M} \psi \Rightarrow \chi [U]$ iff $\models_{M} \neg \psi [U] \vee \models_{M} \chi [U]$ \\
$\models_{M} \phi \Rightarrow \chi [U]$ iff $\models_{M} \neg \phi [U] \vee \models_{M} \chi [U] $  \\
Therefore \\
$\models_{M} \phi \Rightarrow \psi , \psi \Rightarrow \chi [U]$ \\
iff $\models_{M} (\neg \phi \vee \psi \vee \neg \phi \vee \chi) [U]$ \\
iff $\models_{M} (\neg \phi \vee \chi) [U]$ \\
i.e. $\models_{M} \phi \Rightarrow \chi \Box$
}
\end{homeworkSection}

\begin{homeworkSection}
{(e) In what sense is soundness a ''good '' property of  a proof system?}
\problemAnswer{ % This here will give you a box sourrounding your answer.
Because ``sound`` inference gurantees correct conclusions as long as premises are true.
 
}
\end{homeworkSection}

\end{homeworkProblem}



%----------------------------------------------------------------------------------------
%	PROBLEM 3
%----------------------------------------------------------------------------------------

% To have just one problem per page, simply put a \clearpage after each problem

\begin{homeworkProblem}
\vspace{10pt} % Question
    %Your question description here.
    Given that: \\
    1. $(\forall x M(x) \Rightarrow (\exists y  A(y) \wedge H(x,y)))$ \\
    2. $(\forall x M(x) \Rightarrow L(x))$ \\
    3. $(\exists x M(x))$ \\
    Goal: \\
    $L(S) \wedge A(C) \wedge H(S,C)$

\problemAnswer{ % This here will give you a box sourrounding your answer.
    %Your answer inside here.

\begin{tabular} { l l }
    1. $(\forall x M(x) \Rightarrow (\exists y  A(y) \wedge H(x,y)))$ & $\Delta$ \\
    2. $(\forall x M(x) \Rightarrow L(x))$  					& $\Delta$ \\
    3. $(\exists x M(x))$  									& $\Delta$ \\
    4. $M(S) \Rightarrow (\exists y  A(y) \wedge H(S,y))$			& UI, 1 \\
    5. $M(S) \Rightarrow L(S)$								& UI, 2 \\
    6. $M(S)$											& EI, 3 \\
    7. $M(S) \Rightarrow  A(C) \wedge H(S,C)$					& EI, 4 \\
    8. $L(S)$											& MP, 5 \\
    9. $A(C) \wedge H(S,C)$								& MP, 4, 6 \\
    10.$L(S) \wedge A(C) \wedge H(S,C)$ 					& AI, 8 , 9 \\
\end{tabular}
}
\end{homeworkProblem}


%----------------------------------------------------------------------------------------
%	PROBLEM 4
%----------------------------------------------------------------------------------------

% To have just one problem per page, simply put a \clearpage after each problem

\begin{homeworkProblem}
\vspace{10pt} % Question
    %Your question description here.
    Given that: \\
    1. $(\forall x M(x) \Rightarrow (\exists y A(y) \wedge H(x,y)))$ \\
    2. $(\forall x M(x) \Rightarrow L(x))$ \\
    3. $(\forall x L(x) \Rightarrow \neg (\exists y A(y) \wedge H(x,y)))$ \\
    Goal: \\
    $\neg (\exists x M(x))$ \\
\problemAnswer{ % This here will give you a box sourrounding your answer.
    %Your answer inside here.
    \begin{tabular}{ l l}
    1. *Show $\neg (\exists x M(x))$ \\
    2. \ $(\exists x Mx)$ & Assume \\
    3. \ $M(z)$ & 2,EI\\
    4. \ $(\forall x M(x) \Rightarrow (\exists y A(y) \wedge H(x,y)))$ & Prem \\
    5. \ $(\forall x M(x) \Rightarrow L(x))$ & Prem \\
    6. \ $(\forall x L(x) \Rightarrow \neg (\exists y A(y) \wedge H(x,y)))$ &  Prem \\
    7. \ $M(z) \Rightarrow A(c) \wedge H(z,c)$ & 4, UI, EI \\
    8. \ $M(z) \Rightarrow L(k)$ & 5, UI, EI \\
    9. \ $L(K) \Rightarrow \neg (A(c) \wedge H(z,c))$ & 6. UI, EI\\
    11. \ \ *Show $\neg A(c) \wedge H(z,c)$ \\
    12. \ \ $A(c) \wedge H(z,c)$ & 3, 7, MP \\
    13. \ \ $\Box$ \\
    \end{tabular}
}
\end{homeworkProblem}

%----------------------------------------------------------------------------------------
%	PROBLEM 5
%----------------------------------------------------------------------------------------

% To have just one problem per page, simply put a \clearpage after each problem

\begin{homeworkProblem}
\vspace{10pt} % Question
    %Your question description here.
\begin{homeworkSection}
{(a)}
\problemAnswer{ % This here will give you a box sourrounding your answer.
    %Your answer inside here.
}
\end{homeworkSection}

\begin{homeworkSection}
{(b)}
\problemAnswer{ % This here will give you a box sourrounding your answer.
    %Your answer inside here.
}
\end{homeworkSection}

\begin{homeworkSection}
{(c)}
\problemAnswer{ % This here will give you a box sourrounding your answer.
    %Your answer inside here.
}
\end{homeworkSection}

\end{homeworkProblem}


%----------------------------------------------------------------------------------------
%	PROBLEM 6
%----------------------------------------------------------------------------------------

% To have just one problem per page, simply put a \clearpage after each problem

\begin{homeworkProblem}
\vspace{10pt} % Question
    %Your question description here.
\begin{homeworkSection}
{(a) Find the m.g.u. of $\neg Q(x,z,g(x))$ and $Q(f(A), h(y), g(f(y)))$}
\problemAnswer{ % This here will give you a box sourrounding your answer.
    %Your answer inside here.
   1. $x/f(A)$ we get $\neg Q(f(A),z,g(x))$ and $Q(f(A), h(y), g(f(y)))$\\
   2. $z/h(y)$ we get $\neg Q(f(A),h(y),g(x))$ and $Q(f(A), h(y), g(f(y)))$\\
   3. $y/A$ we get $\neg Q(f(A),h(A),g(x))$ and $Q(f(A), h(A), g(f(A)))$\\
   The m.g.u for this problem is 
   \begin{center}
   $((x/f(A)),\ (z/h(y)),\ ($y/A$))$
   \end{center}
}
\end{homeworkSection}

\begin{homeworkSection}
{(b) Find 2 resolvents of $\neg P(x) \vee \neg Q(f(x),y)$ , $Q(u,f(v)) \vee Q(f(u),v)$}
\problemAnswer{ % This here will give you a box sourrounding your answer.
    %Your answer inside here.
   1. r[ 1b, 2b]: $\neg P(u) ,\  Q(f(u), v)$ with m.g.u $((x/u),\ (y/v))$. \\
   2. r[ 1b, 2a]: $\neg P(x), \ Q(f(u), v)$ with m.g.u $((u/f(x),\ (y/f(v)))).$ 
}
\end{homeworkSection}

\begin{homeworkSection}
{(c) Find 2 factors of $Q(f(x),y) \vee Q(y,f(x)) \vee Q(x,f(y))$}
\problemAnswer{ % This here will give you a box sourrounding your answer.
    %Your answer inside here.
   1.f[1 a, c]: $Q(f(x), f(x)) \wedge Q(x,f(f(x)))$ with m.g.u (y/f(x)) \\
   2.f[1 a, b]: $Q(f(x), f(x)) \wedge Q(f(x), f(x))$ with m.g.u (y/f(x))
}
\end{homeworkSection}

\end{homeworkProblem}

%----------------------------------------------------------------------------------------
%	PROBLEM 7
%----------------------------------------------------------------------------------------

% To have just one problem per page, simply put a \clearpage after each problem

\begin{homeworkProblem}
\vspace{10pt} % Question
    %Your question description here.
\begin{homeworkSection}
{(a)}
\problemAnswer{ % This here will give you a box sourrounding your answer.
    %Your answer inside here.
    1. $(\forall x \forall y S(x,y) \Rightarrow (\exists z parent(z,x) \wedge parent(z,y)))$ \\
    2. $(\forall x \forall y M(x,y) \Rightarrow (\exists z P(z,x) \Rightarrow \neg P(z,y)))$ \\
    3. $M(Jocasta,Oedipus)$\\
    4. $S(Jocasta, Oedipus)$ (Reject Goal)
}
\end{homeworkSection}

\begin{homeworkSection}
{(b)}
\problemAnswer{ % This here will give you a box sourrounding your answer.
    %Your answer inside here.
  For 1. $(\forall x \forall y S(x,y) \Rightarrow (\exists z parent(z,x) \wedge parent(z,y)))$ \\
  First we eliminate $\Rightarrow$ \\
  $\exists x \exists \neg S(x,y) \vee (\forall z parent(z,x) \wedge parent(z,y))$ \\
  Then we move $\neg$ inward, which leaves equation the same \\
  Then we skolemize, let $(x/J)(y/O)(z/K)$ \\
  $\neg S(J, O) \vee (P(K, J) \wedge P(K,O))$, distribute $\vee$ over $\wedge$ we have\\
 $(\neg S(J, O) \vee P(K, J)) \wedge (\neg S(J,O) \vee P(K,O))$ \\
  Similarly for 2.$(\forall x \forall y M(x,y) \Rightarrow (\exists z P(z,x) \Rightarrow \neg P(z,y)))$ \\
  we have \\
  $\neg M(J, O) \vee \neg P(K, J) \vee \neg P(K, O)$
  For 3.\\
  $M(J, O)$ \\
  For 4. \\
  $S(J, O)$
}
\end{homeworkSection}

\begin{homeworkSection}
{(c)}
\problemAnswer{ % This here will give you a box sourrounding your answer.
    %Your answer inside here.
    \begin{tabular}{l l}
    1. $(\neg S(J, O) \vee P(K, J)) \wedge (\neg S(J,O) \vee P(K,O))$  & From (b) \\
    2. $\neg M(J, O) \vee \neg P(K, J) \vee \neg P(K, O)$ & From (b)\\
    3. $M(J, O)$ & From (b)\\
    4. $S(J, O)$    & From (b) \\
    5. $\neg P(K, J) \vee \neg P(K, O)$ & r[2a, 3] \\
    6. $P(K, J) \wedge P(K,O)$ & r[1a, 4] \\
    7. $\neg P(K, O)$ & r[5a, 6a] \\
    8. $P(K, O)$  & r[7a 8]\\
    9. $\Box$
    \end{tabular}
}
\end{homeworkSection}

\begin{homeworkSection}
{(d)}
\problemAnswer{ % This here will give you a box sourrounding your answer.
    %Your answer inside here.
    A written solution for this problem is attached at the end of this assignment print-out. 
}
\end{homeworkSection}
\end{homeworkProblem}



\end{document}
