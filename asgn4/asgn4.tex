%%%%%%%%%%%%%%%%%%%%%%%%%%%%%%%%%%%%%%%%%
% Structured General Purpose Assignment
% LaTeX Template
%
% This template has been downloaded from:
% http://www.latextemplates.com
%
% Original author:
% Ted Pavlic (http://www.tedpavlic.com)
%
% Note:
% The \lipsum[#] commands throughout this template generate dummy text
% to fill the template out. These commands should all be removed when 
% writing assignment content.
%
%%%%%%%%%%%%%%%%%%%%%%%%%%%%%%%%%%%%%%%%%

%----------------------------------------------------------------------------------------
%	PACKAGES AND OTHER DOCUMENT CONFIGURATIONS
%----------------------------------------------------------------------------------------

\documentclass{article}


\usepackage{listings}
\lstset{ 
        language=Matlab,                                % choose the language of the code
%       basicstyle=10pt,                                % the size of the fonts that are used for the code
        %numbers=left,                                   % where to put the line-numbers
        %numberstyle=\footnotesize,                      % the size of the fonts that are used for the line-numbers
        %stepnumber=5,                                           % the step between two line-numbers. If it's 1 each line will be numbered
        %numbersep=5pt,                                  % how far the line-numbers are from the code
		%backgroundcolor=\color{light-gray},          % choose the background color. You must add \usepackage{color}
        showspaces=false,                               % show spaces adding particular underscores
        showstringspaces=false,                         % underline spaces within strings
        showtabs=false,                                         % show tabs within strings adding particular underscores
%       frame=single,                                           % adds a frame around the code
%       tabsize=2,                                              % sets default tabsize to 2 spaces
%       captionpos=b,                                           % sets the caption-position to bottom
        breaklines=true,                                        % sets automatic line breaking
        breakatwhitespace=false,                        % sets if automatic breaks should only happen at whitespace
        escapeinside={\%*}{*)},                          % if you want to add a comment within your code
        xleftmargin=.50in
}
\usepackage{color}
\usepackage{fancyhdr} % Required for custom headers
\usepackage{lastpage} % Required to determine the last page for the footer
\usepackage{extramarks} % Required for headers and footers
\usepackage{graphicx} % Required to insert images
\usepackage{lipsum} % Used for inserting dummy 'Lorem ipsum' text into the template
\usepackage{MnSymbol}% Used for \Box






% Margins
\topmargin=-0.45in
\evensidemargin=0in
\oddsidemargin=0in
\textwidth=6.5in
\textheight=9.0in
\headsep=0.25in 

\linespread{1.1} % Line spacing

% Set up the header and footer
\pagestyle{fancy}
\lhead{\hmwkAuthorName} % Top left header
\chead{\hmwkClass\ \hmwkTitle} % Top center header
\rhead{\firstxmark} % Top right header
\lfoot{\lastxmark} % Bottom left footer
\cfoot{} % Bottom center footer
\rfoot{Page\ \thepage\ of\ \pageref{LastPage}} % Bottom right footer
\renewcommand\headrulewidth{0.4pt} % Size of the header rule
\renewcommand\footrulewidth{0.4pt} % Size of the footer rule

\setlength\parindent{0pt} % Removes all indentation from paragraphs

%----------------------------------------------------------------------------------------
%	DOCUMENT STRUCTURE COMMANDS
%	Skip this unless you know what you're doing
%----------------------------------------------------------------------------------------

% Header and footer for when a page split occurs within a problem environment
\newcommand{\enterProblemHeader}[1]{
\nobreak\extramarks{#1}{#1 continued on next page\ldots}\nobreak
\nobreak\extramarks{#1 (continued)}{#1 continued on next page\ldots}\nobreak
}

% Header and footer for when a page split occurs between problem environments
\newcommand{\exitProblemHeader}[1]{
\nobreak\extramarks{#1 (continued)}{#1 continued on next page\ldots}\nobreak
\nobreak\extramarks{#1}{}\nobreak
}

\setcounter{secnumdepth}{0} % Removes default section numbers
\newcounter{homeworkProblemCounter} % Creates a counter to keep track of the number of problems

\newcommand{\homeworkProblemName}{}
\newenvironment{homeworkProblem}[1][Problem \arabic{homeworkProblemCounter}]{ % Makes a new environment called homeworkProblem which takes 1 argument (custom name) but the default is "Problem #"
\stepcounter{homeworkProblemCounter} % Increase counter for number of problems
\renewcommand{\homeworkProblemName}{#1} % Assign \homeworkProblemName the name of the problem
\section{\homeworkProblemName} % Make a section in the document with the custom problem count
\enterProblemHeader{\homeworkProblemName} % Header and footer within the environment
}{
\exitProblemHeader{\homeworkProblemName} % Header and footer after the environment
}

\newcommand{\problemAnswer}[1]{ % Defines the problem answer command with the content as the only argument
\noindent\framebox[\columnwidth][c]{\begin{minipage}{0.98\columnwidth}#1\end{minipage}} % Makes the box around the problem answer and puts the content inside
}

\newcommand{\homeworkSectionName}{}
\newenvironment{homeworkSection}[1]{ % New environment for sections within homework problems, takes 1 argument - the name of the section
\renewcommand{\homeworkSectionName}{#1} % Assign \homeworkSectionName to the name of the section from the environment argument
\subsection{\homeworkSectionName} % Make a subsection with the custom name of the subsection
\enterProblemHeader{\homeworkProblemName\ [\homeworkSectionName]} % Header and footer within the environment
}{
\enterProblemHeader{\homeworkProblemName} % Header and footer after the environment
}
   
%----------------------------------------------------------------------------------------
%	NAME AND CLASS SECTION
%----------------------------------------------------------------------------------------

\newcommand{\hmwkTitle}{Assignment\ 4} % Assignment title
\newcommand{\hmwkDueDate}{Tue,\ Nov.\ 27,\ 2013} % Due date
\newcommand{\hmwkClass}{CS\ 444} % Course/class
\newcommand{\hmwkClassTime}{} % Class/lecture time
\newcommand{\hmwkClassInstructor}{} % Teacher/lecturer
\newcommand{\hmwkAuthorName}{Qiyuan Qiu} % Your name

%----------------------------------------------------------------------------------------
%	TITLE PAGE
%----------------------------------------------------------------------------------------

\title{
\vspace{2in}
\textmd{\textbf{\hmwkClass:\ \hmwkTitle}}\\
\normalsize\vspace{0.1in}\small{Due\ on\ \hmwkDueDate}\\
\vspace{0.1in}\large{\textit{\hmwkClassInstructor\ \hmwkClassTime}}
\vspace{3in}
}

\author{\textbf{\hmwkAuthorName}}
\date{} % Insert date here if you want it to appear below your name

%----------------------------------------------------------------------------------------

\begin{document}

\maketitle

%----------------------------------------------------------------------------------------
%	TABLE OF CONTENTS
%----------------------------------------------------------------------------------------

%\setcounter{tocdepth}{1} % Uncomment this line if you don't want subsections listed in the ToC

%\newpage
%\tableofcontents

\newpage

%----------------------------------------------------------------------------------------
%	PROBLEM 1
%----------------------------------------------------------------------------------------

% To have just one problem per page, simply put a \clearpage after each problem

\begin{homeworkProblem}
\begin{homeworkSection}
{(a)}

\end{homeworkSection}
\end{homeworkProblem}


%----------------------------------------------------------------------------------------
%	PROBLEM 2
%----------------------------------------------------------------------------------------

% To have just one problem per page, simply put a \clearpage after each problem

\begin{homeworkProblem}
\begin{homeworkSection}
{(a)}

(person 
name: Al 
gender: male 
age-LB: 20
age-UP: 20
mother: Mary
father: nil )

(person 
name: Mary 
gender: female 
age-LB: 45
age-UP: 45
mother: Mary
father: nil )

\end{homeworkSection}

\begin{homeworkSection}
{(b)}
\begin{verbatim}
Rule for adding a mother WME to WM, 
IF   (person name:x mother: nil age-LB: y) 
      -(person name: [Mom(x)]) 
THEN ADD (person name: [Mom(x)] gender: female age-LB: [y + 18] mother: nil father: nil)  
      MODIFY 1 (mother: [Mom(x)]) 


Rule for adding a father WME to WM, 
IF  (person name:x father: nil age-LB: y) 
     -(person name: [Dad(x)]) 
THEN ADD (person name: [Dad(x)] gender: male age-LB: [y + 18] mother: nil father: nil) 
     MODIFY 1 (father: [Dad(x)])
\end{verbatim}
\end{homeworkSection}

\begin{homeworkSection}
{(c)}
We can associate with each production rule a number to indicate the frequency this rule has been fired.
Then upon each time that we need to choose a rule to fire, we order all rules by their frequencies.
If we consider only these two rules,  we always check and see if the smaller number of these two rules has passed a threshold (for example 5 (times)). If yes, we will delete these two production rules from our system; otherwise will keep them. \\

Another trick we can apply to this rules is that we have a general rule of thumb that people usually do not live more than 100 years. Base on the observation that every time we add in a new WME by firing either one of these two rules, we will increment the age-LB of that newly introduced person. We can keep an eye on this age-LB, if this { age-LB > 100} we do not fire these two rules. Hence avoid a run-away inference loop.
\end{homeworkSection}

\end{homeworkProblem}

%----------------------------------------------------------------------------------------
%	PROBLEM 3
%----------------------------------------------------------------------------------------

% To have just one problem per page, simply put a \clearpage after each problem

\begin{homeworkProblem}
\begin{homeworkSection}
{(a)}
\begin{verbatim}
(Define STUDENT inherits-from 
           (a PERSON with 
           university-attended (a UNIVERSITY with default UR)
           id-number( a STRING)
           degree-sought(a DEGREE)
           expected-year-of-graduation(a YEAR)
           
\end{verbatim}

\end{homeworkSection}

\begin{homeworkSection}
{(b)}

\problemAnswer{ % This here will give you a box surrounding your answer.

}
\end{homeworkSection}

\begin{homeworkSection}
{(c)}

\problemAnswer{ % This here will give you a box sourrounding your answer.

}
\end{homeworkSection}
\end{homeworkProblem}


\end{document}
